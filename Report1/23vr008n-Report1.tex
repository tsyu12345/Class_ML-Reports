\documentclass{article}[jsarticle]
\usepackage[T1]{fontenc}
\usepackage[dvipdfmx]{hyperref}
\usepackage{lmodern}
\usepackage{latexsym}
\usepackage{amsfonts}
\usepackage{amssymb}
\usepackage{mathtools}
\usepackage{nccmath}
\usepackage{amsthm}
\usepackage{multirow}
\usepackage[dvipdfmx]{graphicx}
\usepackage{wrapfig}
\usepackage{here}
\usepackage{float}
\usepackage{ascmac}
\usepackage{url}

\title{機械学習 課題1}
\author{高林秀 \\ 三宅研究室 博士前期課程1年 \\ V-CampusID : 23vr008n}
\date{\today}

\begin{document}

\maketitle

\begin{abstract}
    \noindent
    本稿は本年度必修授業の機械学習の第1回レポートの答案用紙である。\par
    \noindent
    本稿は、第1回授業~第4回授業までの範囲を対象とし、各回で課された課題に対する解答を記載する。\par
    \noindent
    答案の問題番号は各章のタイトルに記載している。\par
    \noindent
    各問に対する解答は本稿に、コードなどの実行結果は別途GoogleColaboratoryのノートブックに記載した
    巻末の付録から参照できる。
\end{abstract}

\section{第1回授業 : 4/11 宿題1}
    \subsection{問題文}
    二つのデータ点の近さを測る定量的指標の一つとして距離がある。いま、次の三つのデータ点があるとする
    \begin{flalign}
        & \text{データ点1} : (5,2,5.8) \\
        & \text{データ点2} : (7, 10, 1, 12) \\
        & \text{データ点3} : (3, 2, 6, 3)
    \end{flalign}
    上の各データ点は4種類の計測値で与えられている(例えば「緯度、経度、水深、温度」のような感じ)。
    このデータ点同士の間の近さをユークリッド距離で計算し、互いの距離が近いペアの順番を答えよ。データが近いほど距離が小さいことに注意。
    
    \subsection{解答}
    データ点の近さは以下の順番である。
    \begin{enumerate}
        \item データ店1とデータ点3、距離 : $5.4$
        \item データ店1とデータ点2、距離 : $10.0$
        \item データ店2とデータ点3、距離 : $13.6$
    \end{enumerate}
    ユークリッド距離は以下の式で計算できる。
    \begin{flalign*}
        & d(x,y) :\text{データ点xとデータ点yの距離として、}\\
        & d(x,y) = \sqrt{\sum_{i=1}^{n}(x_i - y_i)^2} \\
        & \text{※}x_i,y_i\text{はそれぞれデータ点}x,y\text{のi番目の要素を表す}
    \end{flalign*}
    各点間の距離は以下のように計算できる %TODO:後で手計算すること
    \begin{flalign*}
        & d(1,3) = \sqrt{(5-3)^2 + (2-2)^2 + (5.8-6)^2 + (0-3)^2} = 5.4 \\
        & d(1,2) = \sqrt{(5-7)^2 + (2-10)^2 + (5.8-1)^2 + (0-12)^2} = 10.0 \\
        & d(2,3) = \sqrt{(7-3)^2 + (10-2)^2 + (1-6)^2 + (12-3)^2} = 13.6
    \end{flalign*}

\section{第2回授業 : 4/18 宿題2}
    \subsection{問題文}
    \begin{figure}[H]
        \centering
        \includegraphics[scale=0.7]{./images/MLQ2.png}
    \end{figure}
    \begin{enumerate}
        \item 上の三つの分類モデルそれぞれに対し、混同行列、Precision、Recallを計算しなさい。
        \item モデル1→モデル2→モデル3の順に、Precisionは増加や減少の傾向にあるでしょうか?その傾向はなぜ生じるでしょう?また、Recallについても傾向はどうなっていますか?Precisionの傾向と対比して議論して見てください。
    \end{enumerate}
    
    \subsection{解答:問1}
    まず、各モデルの混同行列とPrecision、Recallを計算した結果は以下のようになる。
    \subsubsection{モデル1}
    % Please add the following required packages to your document preamble:
    % \usepackage{multirow}
    \begin{table}[H]
        \centering
        \begin{tabular}{llcc}
                                                                        & \multicolumn{1}{l|}{}   & \multicolumn{2}{c}{予測}                         \\
                                                                        & \multicolumn{1}{l|}{}   & \multicolumn{1}{c|}{陰性} & 陽性                   \\ \hline
        \multirow{2}{*}{\begin{tabular}[c]{@{}l@{}}実\\ データ\end{tabular}} & \multicolumn{1}{c|}{陰性} & \multicolumn{1}{c|}{1}  & 0                    \\ \cline{2-4} 
                                                                        & \multicolumn{1}{c|}{陽性} & \multicolumn{1}{c|}{6}  & 9                    \\
                                                                        &                         & \multicolumn{1}{l}{}    & \multicolumn{1}{l}{}
        \end{tabular}
        \caption{モデル1の混同行列}
    \end{table}
    \begin{itemize}
        \centering
        \item Precision : $1.0$
        \item Recall : $0.14$
    \end{itemize}

    \subsubsection{モデル2}
    % Please add the following required packages to your document preamble:
    % \usepackage{multirow}
    \begin{table}[H]
        \centering
        \begin{tabular}{llcc}
                                                                        & \multicolumn{1}{l|}{}   & \multicolumn{2}{c}{予測}                         \\
                                                                        & \multicolumn{1}{l|}{}   & \multicolumn{1}{c|}{陰性} & 陽性                   \\ \hline
        \multirow{2}{*}{\begin{tabular}[c]{@{}l@{}}実\\ データ\end{tabular}} & \multicolumn{1}{c|}{陰性} & \multicolumn{1}{c|}{6}  & 3                    \\ \cline{2-4} 
                                                                        & \multicolumn{1}{c|}{陽性} & \multicolumn{1}{c|}{1}  & 6                    \\
                                                                        &                         & \multicolumn{1}{l}{}    & \multicolumn{1}{l}{}
        \end{tabular}
        \caption{モデル2の混同行列}
    \end{table}
    \begin{itemize}
        \centering
        \item Precision : $0.66$
        \item Recall : $0.85$
    \end{itemize}

    \subsubsection{モデル3}
    % Please add the following required packages to your document preamble:
    % \usepackage{multirow}
    \begin{table}[H]
        \centering
        \begin{tabular}{llcc}
                                                                        & \multicolumn{1}{l|}{}   & \multicolumn{2}{c}{予測}                         \\
                                                                        & \multicolumn{1}{l|}{}   & \multicolumn{1}{c|}{陰性} & 陽性                   \\ \hline
        \multirow{2}{*}{\begin{tabular}[c]{@{}l@{}}実\\ データ\end{tabular}} & \multicolumn{1}{c|}{陰性} & \multicolumn{1}{c|}{7}  & 7                    \\ \cline{2-4} 
                                                                        & \multicolumn{1}{c|}{陽性} & \multicolumn{1}{c|}{0}  & 2                    \\
                                                                        &                         & \multicolumn{1}{l}{}    & \multicolumn{1}{l}{}
        \end{tabular}
    \end{table}
    \begin{itemize}
        \centering
        \item Precision : $0.5$
        \item Recall : $1.0$
    \end{itemize}

    \subsection{解答:問2}
    計算結果から、Precision(適合率)は減少の傾向にあるといえる。Precisionは、陽性と予測したデータ数のうち、実際に陽性だったデータ数の割合を示す。計算式で示すと以下の通り。
    \begin{flalign*}
        & Precision = \frac{TP}{TP + FP} \\
        & \text{※}TP\text{は真陽性、FPは偽陽性を表す}
    \end{flalign*}
    この式より、Precisionは分母の、モデルが陽性と予測した数が少なければ値が大きくなる傾向にあることがわかる。
    モデル1では陽性と予測されたデータ数は$1$で、モデル2では$9$、モデル3では$14$と徐々に陽性予測のデータ数が増えており、Precisionの式の分母が大きくなっているので、
    モデル1~モデル3にかけてPrecisionは減少していると考えられる。\par

    また、Recall(再現率)は増加の傾向にあるといえる。Recallは、実際に陽性だったデータのうち、モデルが正しく陽性と予測したデータ数の割合を示す。計算式で示すと以下の通り。
    \begin{flalign*}
        & Recall = \frac{TP}{TP + FN} \\
        & \text{※}TP\text{は真陽性、FNは偽陰性を表す}
    \end{flalign*}
    この式より、Recallは分母の、実際は陽性だが予測では陰性だった割合が少ないほどRecallは大きくなる傾向にあることがわかる。
    実際は陽性だったデータのうち、モデルが陰性と予測したデータの割合がモデル1~モデル3にかけて減少したことから、Recallは増加していると考えられる。\par


\section{第3回授業 : 4/25 宿題1}
    \subsection{問題文}
    Xの数値スケールが大きく(例えば0~1000000)、Yのスケールが小さい(例えば0~1)データに対して、スケーラを使わないで回帰をしたとする。
    \begin{enumerate}
        \item 勾配降下法で学習率を大きくしても小さくしても問題が起こることを説明せよ。
        \item パラメータ$a, b$ごとに学習率を変えて勾配降下法をすれば、スケーラを使わなくても 1. の問題は抑制できるか?具体的な誤差関数の形を使って議論せよ。
    \end{enumerate}

    \subsection{解答}

\section{第4回授業 : }

\section{付録}
\begin{itemize}
    \item GoogleColabノートブック : 
    \item 提出用GoogleDriveフォルダ : 
\end{itemize}
\end{document}